% standard LaTeX packages
\documentclass[12pt,a4paper]{article}
\usepackage[utf8]{inputenc}
\usepackage[T1]{fontenc}
\usepackage[english]{babel}
\usepackage[margin=3cm]{geometry}
\usepackage[hidelinks]{hyperref}

% math packages
\usepackage{mathtools,amsfonts,amssymb,mathdots}

% plotting
\usepackage{tikz}
\usepackage{pgfplots}
\usepackage{caption}

% listings
\usepackage[euler]{textgreek}
\usepackage{xcolor}
\usepackage{listings}
\definecolor{background}{gray}{0.95}
\definecolor{comment}{rgb}{0,0.5,0}
\colorlet{keyword}{blue}
\colorlet{string}{red}
\lstset{numbers=left,
  numberstyle=\tiny,
  breaklines=true,
  tabsize=4,
  morekeywords={with,super,as},
  escapeinside={[$}{$]},
  escapebegin={\color{comment}\begin{math}},
  escapeend={\end{math}},
  frame=single,
  basicstyle=\footnotesize\tt,
  commentstyle=\color{comment},
  keywordstyle=\color{keyword},
  stringstyle=\color{string},
  backgroundcolor=\color{background},
  showstringspaces=false,
  numbers=left,
  numbersep=5pt,
  literate=
    % scandinavian vowels
    {æ}{{\ae}}1
    {å}{{\aa}}1
    {ø}{{\o}}1
    {Æ}{{\AE}}1
    {Å}{{\AA}}1
    {Ø}{{\O}}1
    {é}{{\'e}}1
    {É}{{\'E}}1
    % greek letters
    {α}{{\textalpha}}1
    {β}{{\textbeta}}1
    {γ}{{\textgamma}}1
    {δ}{{\textdelta}}1
    {ε}{{\textepsilon}}1
    {ζ}{{\textzeta}}1
    {η}{{\texteta}}1
    {θ}{{\texttheta}}1
    {ι}{{\textiota}}1
    {κ}{{\textkappa}}1
    {λ}{{\textlambda}}1
    {μ}{{\textmu}}1
    {ν}{{\textnu}}1
    {ξ}{{\textxi}}1
    {ο}{{o}}1
    {π}{{\textpi}}1
    {ρ}{{\textrho}}1
    {σ}{{\textsigma}}1
    {τ}{{\texttau}}1
    {υ}{{\textupsilon}}1
    {φ}{{\textphi}}1
    {ϕ}{{\ensuremath{\phi}}}1
    {χ}{{\textchi}}1
    {ψ}{{\textpsi}}1
    {ω}{{\textomega}}1
    {Α}{{A}}1
    {Β}{{B}}1
    {Γ}{{\textGamma}}1
    {Δ}{{\textDelta}}1
    {Ε}{{E}}1
    {Ζ}{{Z}}1
    {Η}{{H}}1
    {Θ}{{\textTheta}}1
    {Ι}{{I}}1
    {Κ}{{K}}1
    {Λ}{{\textLambda}}1
    {Μ}{{M}}1
    {Ν}{{N}}1
    {Ξ}{{\textXi}}1
    {Ο}{{O}}1
    {∏}{{\textPi}}1
    {Ρ}{{P}}1
    {Σ}{{\textSigma}}1
    {Τ}{{T}}1
    {Υ}{{Y}}1
    {Φ}{{\textPhi}}1
    {Χ}{{X}}1
    {Ψ}{{\textPsi}}1
    {Ω}{{\textOmega}}1
    % miscellaneous
    {°}{{\ensuremath{{}^\circ}}}1
    {²}{{\ensuremath{{}^2}}}1
  }

% bibliography packages
\usepackage[backend=bibtex8,style=authortitle,autocite=footnote,sorting=ynt,dashed=false]{biblatex}

% other packages
\usepackage{filecontents}

% bibliography
\begin{filecontents}{bibliography.bib}
  % ...
\end{filecontents}
\addbibresource{bibliography.bib}

% custom math commands
\newcommand\V[1]{\mathbf{#1}}                  % vector notation
\newcommand\M[1]{\begin{pmatrix} #1 \end{pmatrix}} % matrix shorthand

\begin{document}

\title{Project 1 - FYS3150 Computational Physics}
\author{Fredrik Østrem (\texttt{fredost})}
\date{\today}

\maketitle

\begin{abstract}
  [Abstract goes here]
\end{abstract}

\tableofcontents

\clearpage

\section{Introduction}

\section{Methods}

\subsection{Approximate solution to $-u''(x) = f(x)$}

We consider a differential equation on the form $-u''(x) = f(x)$ with $x \in (0,1)$ and $u(0) = u(1) = 0$, and where $f$ is a known function.

We consider the functions $u$ and $f$ at a sequence of equally spaces points $x_0,x_1,\ldots,x_n$ in the interval $[0,1]$, so that $x_i = ih$ where $h = \frac{1}{n+1}$. We let $v_i = \tilde{u}(x_i) \approx u(x_i)$ be the value of our approximate solution at $x_i$, and let $b_i = h^2 f(x_i)$. Since $u(0) = u(1)$, we have the boundary condition $v_0 = v_{n+1} = 0$.

We can approximate $u''(x)$ by using the Taylor's expansion of $u(x)$, which yields:
\begin{equation}
  u''(x_i) \approx \frac{u(x_i+h) + u(x_i-h) - 2 u(x_i)}{h^2} \approx \frac{v_{i+1} + v_{i-1} - 2 v_i}{h^2}
\end{equation}
which, when subsituted into our differential equation gives
\begin{equation}
  2v_i - v_{i+1} - v_{i-1} = b_i
\end{equation}

Since this is a linear equation with $v_{i-1},v_i,v_{i+1}$ as unknowns, we can write this as
\begin{equation}
  \underbrace{\M{ 0 & 0 & \cdots & -1 & 2 & -1 & \cdots & 0 & 0 }}_{\V{a}_i}
  \M{ v_0 \\ v_1 \\ \vdots \\ v_{i-1} \\ v_i \\ v_{i+1} \\ \vdots \\ v_{n-1} \\ v_n }
  = b_i
\end{equation}
We can do this for every index $i$, and $\V{a}_i$ is shifted a single step to the right compared to $\V{a}_{i-1}$. Therefore, when we
look at all values of $i = 1, \ldots, n$, we get the tridiagonal matrix $A$:
\begin{equation}
  A = \M{ \V{a}_0 \\ \vdots \\ \V{a}_{i-1} \\ \V{a}_i \\ \V{a}_{i+1} \\ \vdots \\ \V{a}_n }
    = \M{
      2  & -1 & 0 & \cdots &         & 0 \\
      -1 & 2  & -1 & 0 & \iddots & \\
      0 & -1 & 2  & \ddots & 0 & \vdots \\
      \vdots & 0 & \ddots & 2 & -1 & 0\\
        & \iddots & 0 & -1 & 2 & -1 \\
      0 & & \cdots & 0 & -1 & 2
    }
\end{equation}
such that $A \V{v} = \V{b}$.

% f(x) = 100e^{-10x}

\subsection{General algorithm for solving tridiagonal matrix equations}
\label{sec:b}

In general, we can write an $n \times n$ tridiagonal matrix $A$ as
\begin{equation}
  A = \M{
    b_1 & c_1 & 0 & \ldots & \ldots & \ldots \\
    a_1 & b_2 & c_2 & \ldots & \ldots & \ldots \\
    & a_2 & b_3 & c_3 & \ldots & \ldots \\
    & \ldots & \ldots & \ldots & \ldots & \ldots \\
    & & & a_{n-2} & b_{n-1} & c_{n-1} \\
    & & & & a_{n-1} & b_n
  }
\end{equation}

We can solve the matrix equation $A \V{v} = \V{b}$ in three steps:
\begin{enumerate}
  \item Eliminate the lower diagonal $a_1,a_2,\ldots,a_{n-1}$ through forward substitution.
  \item Eliminate the upper diagonal $c_1,c_2,\ldots,c_{n-1}$ through backward substitution.
  \item Divide each row $i$ by $b_i$ to get only $1$ elements along the main diagonal.
\end{enumerate}
When doing these steps, we transform $A$ into the identity matrix, and transform $\V{b}$ into $\V{v}$. (See section \ref{apx:b} in the appendix for the implementation of this algorithm in C++). We can see the results of the approximation in figure \ref{fig:b1}.

\begin{figure}[!ht]
  \begin{tikzpicture}
    \begin{axis}[
        width=\textwidth,
        height=0.5\textwidth,
        xmin=0, xmax=1,
        xlabel=$x$,
        ymin=0, ymax=2,
        ylabel=$u(x)$,
        legend cell align=length,
      ]
      \addplot[black,thick,domain=0:1,samples=100] {1 - (1 - exp(-10))*x - exp(-10*x)};
      \addplot[red] table[x=x,y=v] {plots_b_10.dat};
      \addplot[red!50!blue] table[x=x,y=v] {plots_b_100.dat};
      \addplot[blue] table[x=x,y=v] {plots_b_1000.dat};

      \legend{Closed-form solution,$N = 10$,$N = 100$,$N = 1000$}
    \end{axis}
  \end{tikzpicture}

  \caption{Plot of approximate solutions based on the method described in section \ref{sec:b}} \label{fig:b1}
\end{figure}

As we see in the plots, the solution shape of the solution is about right for all three approximations, but for low $N$ it's a bit off from its closed-form solution. However, we see that for large $N$, the numerical approximations gets closer to the actual solution, and by $N = 1000$ it's hard to differentite between them in the plot.

From the implementation, we can see that the first step uses 6 floating-point operations for every row except the first, the second step uses 5 for every row except the last, and the third step uses 2 for every row; in total, the number of floating-point operations is $6 (N - 1) + 5 (N - 1) + 2 N = 13N - 11$.

\section{Results}

\section{Conclusions and perspectives}

\clearpage
\appendix
\section{Appendix}

\subsection{Algorithm for solving tridiagonal matrices}
\label{apx:b}

\lstinputlisting[language=C++]{main.cc}

\clearpage

\printbibliography[heading=bibnumbered,title=Bibliography]

\end{document}
