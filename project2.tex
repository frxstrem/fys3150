% standard LaTeX packages
\documentclass[11pt,a4paper]{article}
\usepackage[utf8]{inputenc}
\usepackage[T1]{fontenc}
\usepackage[english]{babel}
\usepackage[margin=3cm]{geometry}
\usepackage[hidelinks]{hyperref}
\usepackage{parskip}
\usepackage{xifthen}

% math packages
\usepackage{mathtools,amsfonts,amssymb,mathdots}
\usepackage{siunitx}
\mathtoolsset{showonlyrefs}

% plotting and tables
\usepackage{tikz}
\usepackage{pgfplots}
\usepackage{pgfplotstable}
\usepackage{caption}
\pgfplotsset{compat=1.13}

% listings
\usepackage[euler]{textgreek}
\usepackage{xcolor}
\usepackage{listings}
\definecolor{background}{gray}{0.95}
\definecolor{comment}{rgb}{0,0.5,0}
\colorlet{keyword}{blue}
\colorlet{string}{red}
\lstset{numbers=left,
  numberstyle=\tiny,
  breaklines=true,
  tabsize=4,
  morekeywords={with,super,as},
  escapeinside={[$}{$]},
  escapebegin={\color{comment}\begin{math}},
  escapeend={\end{math}},
  frame=single,
  basicstyle=\footnotesize\tt,
  commentstyle=\color{comment},
  keywordstyle=\color{keyword},
  stringstyle=\color{string},
  backgroundcolor=\color{background},
  showstringspaces=false,
  numbers=left,
  numbersep=5pt,
  literate=
    % scandinavian vowels
    {æ}{{\ae}}1
    {å}{{\aa}}1
    {ø}{{\o}}1
    {Æ}{{\AE}}1
    {Å}{{\AA}}1
    {Ø}{{\O}}1
    {é}{{\'e}}1
    {É}{{\'E}}1
    % greek letters
    {α}{{\textalpha}}1
    {β}{{\textbeta}}1
    {γ}{{\textgamma}}1
    {δ}{{\textdelta}}1
    {ε}{{\textepsilon}}1
    {ζ}{{\textzeta}}1
    {η}{{\texteta}}1
    {θ}{{\texttheta}}1
    {ι}{{\textiota}}1
    {κ}{{\textkappa}}1
    {λ}{{\textlambda}}1
    {μ}{{\textmu}}1
    {ν}{{\textnu}}1
    {ξ}{{\textxi}}1
    {ο}{{o}}1
    {π}{{\textpi}}1
    {ρ}{{\textrho}}1
    {σ}{{\textsigma}}1
    {τ}{{\texttau}}1
    {υ}{{\textupsilon}}1
    {φ}{{\textphi}}1
    {ϕ}{{\ensuremath{\phi}}}1
    {χ}{{\textchi}}1
    {ψ}{{\textpsi}}1
    {ω}{{\textomega}}1
    {Α}{{A}}1
    {Β}{{B}}1
    {Γ}{{\textGamma}}1
    {Δ}{{\textDelta}}1
    {Ε}{{E}}1
    {Ζ}{{Z}}1
    {Η}{{H}}1
    {Θ}{{\textTheta}}1
    {Ι}{{I}}1
    {Κ}{{K}}1
    {Λ}{{\textLambda}}1
    {Μ}{{M}}1
    {Ν}{{N}}1
    {Ξ}{{\textXi}}1
    {Ο}{{O}}1
    {∏}{{\textPi}}1
    {Ρ}{{P}}1
    {Σ}{{\textSigma}}1
    {Τ}{{T}}1
    {Υ}{{Y}}1
    {Φ}{{\textPhi}}1
    {Χ}{{X}}1
    {Ψ}{{\textPsi}}1
    {Ω}{{\textOmega}}1
    % miscellaneous
    {°}{{\ensuremath{{}^\circ}}}1
    {²}{{\ensuremath{{}^2}}}1
  }

% bibliography packages
\usepackage[backend=bibtex8,style=numeric,autocite=inline,sorting=ynt]{biblatex}

% other packages
\usepackage{filecontents}

% bibliography
\begin{filecontents}{bibliography.bib}
  @online{project2,
    author   = {Computational Physics I FYS3150},
    title    = {Project 2},
    year     = {2016},
    url      = {https://github.com/CompPhysics/ComputationalPhysics/blob/master/doc/Projects/2016/Project2/project2_2016.pdf},
  }

  @article{armadillo,
    author   = {Conrad Sanderson and Ryan Curtin.},
    title    = {Armadillo: a template-based C++ library for linear algebra.},
    journal  = {Journal of Open Source Software},
    year     = 2016,
    volume   = 1,
    pages    = {26},
  }
\end{filecontents}
\DeclareFieldFormat[article]{volume}{\bibstring{jourvol}\addnbspace #1}
\addbibresource{bibliography.bib}
\nocite{*}

% custom math commands
\newcommand\V[1]{\mathbf{#1}}                  % vector notation
\newcommand\M[1]{\begin{bmatrix} #1 \end{bmatrix}} % matrix shorthand

% commands for derivatives/integral dx expressions
\newcommand\D[1]{{\,\mathrm{d}{#1}}}
\newcommand\dd[2][]{{%
  \ifthenelse{\isempty{#1}}%
    {\mathrm{d}{#2}}%
    {\mathrm{d}^{#1}{#2}}%
}}
\newcommand\df[3][]{{%
  \ifthenelse{\isempty{#1}}%
    {\frac{\mathrm{d}{#2}}{\mathrm{d}{#3}}}%
    {\frac{\mathrm{d}^{#1}{#2}}{{\mathrm{d}{#3}^{#1}}}}%
}}

\begin{document}

\title{Project 2 - FYS3150 Computational Physics}
\author{Fredrik Østrem (\texttt{fredost}) \\ Joseph Knutson (\texttt{josephkn})}
\date{\today}

\maketitle

\begin{abstract}
  \color{red}\textit{INSERT ABSTRACT HERE}
\end{abstract}

\tableofcontents

\clearpage

% Your task is to solve this
% equation by reformulating it in a discretized form as an eigenvalue
% equation to be solved with Jacobi's method. To achieve this you will
% have to write your own code which implements Jacobi's method.

\section{Introduction}
Electrons confined in small places, such as in semiconductors (width of 20nm), are a hot topic in modern solid state physics.
Their interactions are therefore a relevant field of study for today's physicists.
In this project we are going to solve Schroedinger's equation for two
electrons in a three-dimensional harmonic oscillator well. This will be done with and
without a repulsive Coulomb potential while we assume spherical symmetry.

Here we present the solution of the radial part of Schroedinger's equation for one electron
\begin{equation}
  -\frac{\hbar^2}{2 m} \left ( \frac{1}{r^2} \frac{d}{dr} r^2
  \frac{d}{dr} - \frac{l (l + 1)}{r^2} \right )R(r)
     + V(r) R(r) = E R(r).
\end{equation}
In our case $V(r)$ is the harmonic oscillator potential $(1/2)kr^2$ with
$k=m\omega^2$ and $E$ is
the energy of the harmonic oscillator in three dimensions.
The oscillator frequency is $\omega$ and the energies are

\begin{equation}
E_{nl}=  \hbar \omega \left(2n+l+\frac{3}{2}\right),
\end{equation}
with $n=0,1,2,\dots$ and $l=0,1,2,\dots$.
$r\in [0,\infty)$ since we're using spherical coordinates.
The quantum number
$l$ is the orbital momentum of the electron.

We substitute $R(r) = (1/r) u(r)$, set $l = 0$ and obtain
\begin{equation}
  -\frac{\hbar^2}{2 m} \frac{d^2}{dr^2} u(r)
       +  V(r)u(r)  = E u(r) .
\end{equation}
The boundary conditions are $u(0)=0$ and $u(\infty)=0$.

Now we introduce a dimensionless variable $\rho = (1/\alpha) r$
where $\alpha$ is a constant with dimension length. We do this to remove uneccesary factors later. We also set $V(\rho)$ equal to the HO potential$(1/2) k \alpha^2\rho^2$ and rewrite our equation:

\begin{equation}
  -\frac{d^2}{d\rho^2} u(\rho)
       + \frac{mk}{\hbar^2} \alpha^4\rho^2u(\rho)  = \frac{2m\alpha^2}{\hbar^2}E u(\rho) .
\end{equation}
The constant $\alpha$ can now be fixed
so that

\begin{equation}
\frac{mk}{\hbar^2} \alpha^4 = 1,
\end{equation}
or

\begin{equation}
\alpha = \left(\frac{\hbar^2}{mk}\right)^{1/4}.
\end{equation}
Defining

\begin{equation}
\lambda = \frac{2m\alpha^2}{\hbar^2}E,
\end{equation}
we can rewrite Schroedinger's equation as

\begin{equation}
  -\frac{d^2}{d\rho^2} u(\rho) + \rho^2u(\rho)  = \lambda u(\rho) .
\end{equation}
This is the first equation to solve numerically. In three dimensions
the eigenvalues for $l=0$ are
$\lambda_0=3,\lambda_1=7,\lambda_2=11,\dots .$

We use the by now standard
expression for the second derivative of a function $u$
\begin{equation}
    u''=\frac{u(\rho+h) -2u(\rho) +u(\rho-h)}{h^2} +O(h^2),
    \label{eq:diffoperation}
\end{equation}
where $h$ is our step.
Next we define minimum and maximum values for the variable $\rho$,
$\rho_{\mathrm{min}}=0$  and $\rho_{\mathrm{max}}$, respectively.
You need to check your results for the energies against different values
$\rho_{\mathrm{max}}$, since we cannot set
$\rho_{\mathrm{max}}=\infty$.

With a given number of mesh points, $N$, we
define the step length $h$ as, with $\rho_{\mathrm{min}}=\rho_0$  and $\rho_{\mathrm{max}}=\rho_N$,

\begin{equation*}
  h=\frac{\rho_N-\rho_0 }{N+1}.
\end{equation*}
The value of $\rho$ at a point $i$ is then
\[
    \rho_i= \rho_0 + ih \hspace{1cm} i=0,1,2,\dots , N.
\]
We can rewrite the Schroedinger equation for a value $\rho_i$ as

\[
-\frac{u(\rho_i+h) -2u(\rho_i) +u(\rho_i-h)}{h^2}+\rho_i^2u(\rho_i)  = \lambda u(\rho_i),
\]
or in  a more compact way

\[
-\frac{u_{i+1} -2u_i +u_{i-1}}{h^2}+\rho_i^2u_i=-\frac{u_{i+1} -2u_i +u_{i-1} }{h^2}+V_iu_i  = \lambda u_i,
\]
where $V_i=\rho_i^2$ is the harmonic oscillator potential.

We define first the diagonal matrix element
\begin{equation*}
   d_i=\frac{2}{h^2}+V_i,
\end{equation*}
and the non-diagonal matrix element
\begin{equation*}
   e_i=-\frac{1}{h^2}.
\end{equation*}
In this case the non-diagonal matrix elements are given by a mere constant.
\emph{All non-diagonal matrix elements are equal}.
With these definitions the Schroedinger equation takes the following form

\begin{equation*}
d_iu_i+e_{i-1}u_{i-1}+e_{i+1}u_{i+1}  = \lambda u_i,
\end{equation*}
where $u_i$ is unknown. We can write the
latter equation as a matrix eigenvalue problem
\begin{equation}
\end{equation}
Since the values of $u$ at the two endpoints are known via the boundary conditions, we can skip the rows and columns that involve these values. Inserting the values for $d_i$ and $e_i$ we have the following matrix
\begin{equation}
    \begin{bmatrix} \frac{2}{h^2}+V_1 & -\frac{1}{h^2} & 0   & 0    & \dots  &0     & 0 \\
                                -\frac{1}{h^2} & \frac{2}{h^2}+V_2 & -\frac{1}{h^2} & 0    & \dots  &0     &0 \\
                                0   & -\frac{1}{h^2} & \frac{2}{h^2}+V_3 & -\frac{1}{h^2}  &0       &\dots & 0\\
                                \dots  & \dots & \dots & \dots  &\dots      &\dots & \dots\\
                                0   & \dots & \dots & \dots  &-\frac{1}{h^2}  &\frac{2}{h^2}+V_{N-2} & -\frac{1}{h^2}\\
                                0   & \dots & \dots & \dots  &\dots       &-\frac{1}{h^2} & \frac{2}{h^2}+V_{N-1}
             \end{bmatrix}
\label{eq:matrixse}
\end{equation}
\section{Methods}

\subsection{}
% (a)

Assume that $\V{U}$ is an orthogonal transformation. Then by definition, $\V{U}^T \V{U} = \V{I}$,
so for any vectors $\V{u}$ and $\V{v}$ we see that
\begin{equation}
  \left( \V{U} \V{u} \right) \cdot \left( \V{U} \V{v} \right)
    = \left( \V{U} \V{u} \right)^T \left( \V{U} \V{v} \right)
    = \V{u}^T \underbrace{\V{U}^T \V{U}}_{= \, \V{I}} \V{v}
    = \V{u}^T \V{v} = \V{u} \cdot \V{v}
\end{equation}

Since the dot product of vectors are preserved by $\V{U}$, orthogonality (that is, the dot product
being zero) is also preserved.

\subsection{}
% (b)

We have the differential equation
\begin{equation}
  - \df[2]{}{\rho} u(\rho) + V(\rho) u(\rho) = \lambda u(\rho)
\end{equation}
where $V(\rho) = \rho^2$, that we want to solve. By using Taylor expansion of $u(\rho)$ and discretizing $u$ and $\rho$ with step length $h$, we
get the equation
\begin{equation}
  - \frac{1}{h^2} u_{i-1} + \left( \frac{2}{h^2} - V_i \right) u_i - \frac{1}{h^2} u_{i+1} = \lambda u_i
\end{equation}

We can write this as a matrix eigenvector equation, $\V{A} \V{u} = \lambda \V{u}$, where
\begin{equation}
  \V{A} = \M{
    d_0 & e_0 & 0   & 0    & \dots  &0     & 0 \\
    e_1 & d_1 & e_1 & 0    & \dots  &0     &0 \\
    0   & e_2 & d_2 & e_2  &0       &\dots & 0\\
    \dots  & \dots & \dots & \dots  &\dots      &\dots & \dots\\
    0   & \dots & \dots & \dots  &\dots  e_{N-1}     &d_{N-1} & e_{N-1}\\
    0   & \dots & \dots & \dots  &\dots       &e_{N} & d_{N}
  } , \quad
  d_i = \frac{2}{h^2} - V_i , \quad
  e_i = - \frac{1}{h^2}
\end{equation}

We can solve this equation by finding an orthonormal matrix $\V{S}$ and a diagonal matrix $\V{B}$
such that
\begin{equation}
  \V{A} = \V{S}^T \V{B} \V{S}
\end{equation}
where the diagonal elements of $B$, $b_{ii}$, are the eigenvalues $\lambda_i$ of $A$, and the row
vectors $\V{s}_i$ of $\V{S}$ are the eigenvectors of $\V{A}$. We can see this since, if $\V{y} = \V{e}_i$, then $\V{y}$ is an eigenvector of $\V{B}$ with eigenvalue $\lambda_i$; if then $\V{x} = \V{S}^T \V{y}$, then
\begin{equation}
  \V{A} \V{x}
    = \V{S}^T \V{B} \V{S} \V{x}
    = \V{S}^T \V{B} \V{S} \V{S}^T \V{y}
    = \V{S}^T \V{B} \V{y}
    = \V{S}^T \lambda_i \V{y}
    = \lambda_i \, \V{S}^T \V{y}
    = \lambda_i \V{x}
\end{equation}
where $\V{x} = \V{S}^T \V{e}_i$ is the $i$'th column of $\V{S}^T$, and thus also the $i$'th row of $\V{S}$.

We'll be using the Jacobi method, in which we in multiple steps will remove the off-diagonal elements
of $A$ with orthonormal transformations, until we have a matrix where the off-diagonal elements are
sufficiently small. For each step, we find the element $a_{kl}$ that is the largest off-diagonal
element in terms of absolute value.

\section{Results}

\section{Conclusions}

\clearpage
\appendix
\section{Appendix}

All files used in this project can be found at \url{https://github.com/frxstrem/fys3150/tree/master/project2}.
The following code files are used:
\begin{itemize}
  \item {\it \color{red} none}
\end{itemize}

\clearpage

\printbibliography[heading=bibnumbered,title=Bibliography]

\end{document}
